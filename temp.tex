\hypertarget{projecte-dintercanvi-de-llibres}{%
\section{Projecte d'Intercanvi de
Llibres}\label{projecte-dintercanvi-de-llibres}}

\hypertarget{descripciuxf3-del-projecte}{%
\subsection{1. Descripció del
Projecte}\label{descripciuxf3-del-projecte}}

Aquest projecte implementa una aplicació web utilitzant Django que
permet als usuaris intercanviar, vendre o donar llibres entre ells
mitjançant un sistema de punts.

\hypertarget{model-relacional}{%
\subsection{2. Model Relacional}\label{model-relacional}}

El model de dades consisteix en les següents entitats i relacions:

\hypertarget{entitats-principals}{%
\subsubsection{Entitats Principals}\label{entitats-principals}}

Declarades en \texttt{models.py}: 1. \textbf{User} (Usuari) - Estén el
model d'usuari de Django (AbstractUser) - Incorpora gestió
d'autenticació nativa de Django - Camps addicionals: - \texttt{points}:
Punts disponibles per a transaccions - \texttt{location}: Ubicació de
l'usuari - \texttt{joined\_date}: Data de registre

\begin{enumerate}
\def\labelenumi{\arabic{enumi}.}
\setcounter{enumi}{1}
\tightlist
\item
  \textbf{Book} (Llibre)

  \begin{itemize}
  \tightlist
  \item
    Identificat per ISBN (clau primària)
  \item
    Inclou informació bibliogràfica com títol, autor, tema
  \item
    Preu base en punts
  \end{itemize}
\item
  \textbf{Review} (Ressenya)

  \begin{itemize}
  \tightlist
  \item
    Permet als usuaris opinar sobre els llibres
  \item
    Cada usuari només pot fer una ressenya per llibre
  \item
    Té com a claus forana \texttt{user} (usuari que fa la ressenya) i
    \texttt{book} (llibre ressenyat)
  \end{itemize}
\end{enumerate}

\hypertarget{relacions-entre-entitats}{%
\subsubsection{Relacions entre
Entitats}\label{relacions-entre-entitats}}

\begin{enumerate}
\def\labelenumi{\arabic{enumi}.}
\tightlist
\item
  \textbf{Have} (Tenir)

  \begin{itemize}
  \tightlist
  \item
    Relació molts-a-molts entre User i Book
  \item
    Indica quins llibres té cada usuari disponibles per intercanvi/venda
  \end{itemize}
\item
  \textbf{Want} (Voler)

  \begin{itemize}
  \tightlist
  \item
    Relació molts-a-molts entre User i Book
  \item
    Utilitza claus foranes per referenciar llibres i usuaris
    (\texttt{models.ForeignKey})
  \item
    Indica quins llibres desitja cada usuari
  \item
    Inclou un camp de prioritat
  \item
    Restricció: Un usuari no pot voler el mateix llibre més d'una vegada
  \end{itemize}
\item
  \textbf{SaleDonation} (VendaDonació)

  \begin{itemize}
  \tightlist
  \item
    Registra la venda o donació d'un llibre d'un usuari
  \item
    Inclou preu en punts, ubicació i estat de la transacció
  \end{itemize}
\item
  \textbf{Exchange} (Intercanvi)

  \begin{itemize}
  \tightlist
  \item
    Registra intercanvis de llibres entre dos usuaris
  \item
    Inclou llibres intercanviats, ubicació i estat de la transacció
  \end{itemize}
\end{enumerate}

Respecte al diagrama original s'ha respectat. \#\#\# Diagrama de
Relacions

\begin{verbatim}
User 1---* Review *---1 Book
User 1---* Have *---1 Book
User 1---* Want *---1 Book
User 1---* SaleDonation *---1 Book
User(User1) 1---* Exchange *---1 User(User2)
Exchange *---1 Book(Book1)
Exchange *---1 Book(Book2)
\end{verbatim}

\includegraphics{https://i.imgur.com/enuaXcb.png}

Les relacions estan dissenyades per poder-se referenciar mitjançant
claus foranes, i inclouen camps \texttt{related\_name} per facilitar les
consultes en ambdues direccions.

\begin{quote}
Tot i que encara no s'ha implementat, s'ha decidit que només es guardarà
a la base de dades pròpia de l'aplicació els llibres que els usuaris
hagin intercanviat, venut, donat, volgut o valorat. La resta de llibres
i la seva informació es consultaran a través d'API externes.
\end{quote}

\hypertarget{tauler-dadministraciuxf3}{%
\subsection{3. Tauler d'Administració}\label{tauler-dadministraciuxf3}}

S'ha creat i activat un tauler d'administració per gestionar les
entitats del sistema. El tauler permet als administradors gestionar
usuaris, llibres, ressenyes i transaccions, a més s'han implementat
filtres i funcionalitats de cerca per facilitar la gestió de les dades.

\begin{enumerate}
\def\labelenumi{\arabic{enumi}.}
\tightlist
\item
  \textbf{Activació del tauler}:

  \begin{itemize}
  \tightlist
  \item
    S'ha activat el mòdul d'administració incloent
    \texttt{django.contrib.admin} a \texttt{INSTALLED\_APPS} al fitxer
    \texttt{settings.py}
  \item
    S'ha registrat al fitxer \texttt{urls.py} principal amb la ruta
    \texttt{/admin/}
  \end{itemize}
\item
  \textbf{Personalització dels models}:

  \begin{itemize}
  \tightlist
  \item
    S'ha creat un fitxer \texttt{admin.py} a l'aplicació web on es
    registren i configuren tots els models
  \item
    Cada model disposa d'una classe Admin específica que hereta de
    \texttt{admin.ModelAdmin}
  \end{itemize}
\end{enumerate}

El tauler d'administració permet:

\begin{enumerate}
\def\labelenumi{\arabic{enumi}.}
\tightlist
\item
  \textbf{Gestió d'Usuaris}:

  \begin{itemize}
  \tightlist
  \item
    Classe personalitzada \texttt{CustomUserAdmin} que estén
    \texttt{UserAdmin}
  \item
    Visualització i edició dels camps estàndards i personalitzats
    (punts, ubicació, data d'inscripció)
  \item
    Filtres
  \end{itemize}
\item
  \textbf{Gestió de Llibres}:

  \begin{itemize}
  \tightlist
  \item
    Visualització i edició de tots els detalls bibliogràfics
  \item
    Filtres per tema i data de publicació
  \item
    Cerca per ISBN, títol, autor i tema
  \end{itemize}
\item
  \textbf{Gestió de Relacions}:

  \begin{itemize}
  \tightlist
  \item
    Interfícies intuïtives per a les relacions Have, Want, Exchange i
    SaleDonation
  \item
    Visualització de les valoracions (Reviews) amb filtres per puntuació
  \end{itemize}
\item
  \textbf{Funcionalitats generals}:

  \begin{itemize}
  \tightlist
  \item
    Filtres per camps rellevants a cada model
  \item
    Camps de cerca per facilitar la localització ràpida d'entitats
  \item
    Organització jeràrquica per dates en els models que ho requereixen
  \end{itemize}
\end{enumerate}

\textbf{Accés al Tauler d'Administració}

L'administració està disponible a la URL \texttt{/admin/} i requereix
credencials de superusuari. Es pot crear un superusuari mitjançant:

\begin{Shaded}
\begin{Highlighting}[]
\ExtensionTok{python}\NormalTok{ manage.py createsuperuser}
\end{Highlighting}
\end{Shaded}

O dins del contenidor Docker:

\begin{Shaded}
\begin{Highlighting}[]
\ExtensionTok{docker{-}compose}\NormalTok{ exec web python manage.py createsuperuser}
\end{Highlighting}
\end{Shaded}

Es podrà accedir al tauler d'administració amb les credencials del
superusuari creat al navegador web a la URL
\texttt{http://localhost:8000/admin/}.

\hypertarget{sistema-dautenticaciuxf3-i-registre}{%
\subsection{4. Sistema d'Autenticació i
Registre}\label{sistema-dautenticaciuxf3-i-registre}}

S'ha implementat un sistema d'autenticació basat estenent el sistema
d'usuari de Django i el seu sistema de formularis. El sistema permet als
usuaris registrar-se, iniciar sessió i gestionar el seu perfil d'usuari.
D'aquesta manera s'aprofita les funcionalitats de validació que els
camps ja porten per defecte.

\hypertarget{formularis-dautenticaciuxf3}{%
\subsubsection{Formularis
d'Autenticació}\label{formularis-dautenticaciuxf3}}

\begin{enumerate}
\def\labelenumi{\arabic{enumi}.}
\tightlist
\item
  \textbf{Formulari de Registre (\texttt{CustomUserCreationForm})}:

  \begin{itemize}
  \tightlist
  \item
    Estén el \texttt{UserCreationForm} natiu de Django
  \item
    Afegeix camps addicionals: correu electrònic (obligatori) i ubicació
    (opcional)
  \item
    Utilitza transaccions atòmiques per garantir la consistència de les
    dades
  \item
    Gestiona automàticament la creació del perfil d'usuari personalitzat
  \end{itemize}
\item
  \textbf{Formulari d'Inici de Sessió (\texttt{LoginForm})}:

  \begin{itemize}
  \tightlist
  \item
    Formulari personalitzat per a la validació de credencials
  \item
    Camp de nom d'usuari i contrasenya amb validació adequada
  \end{itemize}
\end{enumerate}

\hypertarget{procuxe9s}{%
\subsubsection{Procés}\label{procuxe9s}}

\begin{enumerate}
\def\labelenumi{\arabic{enumi}.}
\tightlist
\item
  \textbf{Procés de Registre}:

  \begin{itemize}
  \item
    Formulari amb validació completa de contrasenyes i unicitat
    d'usuaris
  \item
    Assignació d'ubicació si està disponible
  \end{itemize}
\item
  \textbf{Procés d'Inici de Sessió}:

  \begin{itemize}
  \tightlist
  \item
    Validació segura de credencials
  \item
    Redirecció personalitzada després de l'autenticació
  \end{itemize}
\end{enumerate}

\begin{quote}
Utilitzar com base els formularis de Django per a la creació d'usuaris i
autenticació ens ha permès aprofitar les funcionalitats de validació i
no haver-les de implementar manualment.
\end{quote}

\hypertarget{configuraciuxf3-de-docker}{%
\subsection{5. Configuració de Docker}\label{configuraciuxf3-de-docker}}

El projecte s'ha configurat per funcionar en un entorn containeritzat
utilitzant Docker, de manera que es pot desplegar fàcilment en qualsevol
màquina amb Docker instal·lat.

\hypertarget{estructura-de-contenidors}{%
\subsubsection{Estructura de
contenidors}\label{estructura-de-contenidors}}

L'entorn Docker consta principalment d'un contenidor web que executa
l'aplicació Django:

\begin{enumerate}
\def\labelenumi{\arabic{enumi}.}
\tightlist
\item
  \textbf{Contenidor web}: Basat en Python, amb totes les dependències
  necessàries per executar l'aplicació Django
\item
  \textbf{Base de dades}: Actualment utilitzem SQLite (inclosa dins del
  contenidor web) \textgreater{} \emph{Nota: De moment no s'ha
  implementat una base de dades externa com PostgreSQL, depenent de les
  necessitats futures del projecte ja es valorarà si és necessari
  fer-ho.}
\end{enumerate}

\hypertarget{arxius-de-configuraciuxf3}{%
\subsubsection{Arxius de configuració}\label{arxius-de-configuraciuxf3}}

El projecte inclou els següents arxius de configuració de Docker:

\begin{itemize}
\tightlist
\item
  \textbf{Dockerfile}: Defineix la imatge base, instal·la dependències i
  configura l'entorn d'execució
\item
  \textbf{docker-compose.yml}: Orquestra els serveis, defineix els
  volums per persistència de dades i configura les variables d'entorn
\end{itemize}

\hypertarget{volums-i-persistuxe8ncia}{%
\subsubsection{Volums i persistència}\label{volums-i-persistuxe8ncia}}

S'han configurat volums per garantir la persistència de les dades: - El
codi font del projecte es munta com un volum al contenidor - La base de
dades SQLite es manté persistent entre execucions

\hypertarget{execuciuxf3-del-projecte-amb-docker}{%
\subsubsection{Execució del Projecte amb
Docker}\label{execuciuxf3-del-projecte-amb-docker}}

Per executar l'aplicació en un entorn local:

\begin{enumerate}
\def\labelenumi{\arabic{enumi}.}
\item
  \textbf{Prerequisits}:

  \begin{itemize}
  \tightlist
  \item
    Docker i Docker Compose instal·lats al sistema
  \item
    Git per clonar el repositori
  \end{itemize}
\item
  \textbf{Clonar el repositori}:

\begin{Shaded}
\begin{Highlighting}[]
\FunctionTok{git}\NormalTok{ clone https://github.com/Krisoc123/ProjecteWeb.git}
\BuiltInTok{cd}\NormalTok{ ProjecteWeb}
\end{Highlighting}
\end{Shaded}
\item
  \textbf{Construir i iniciar els contenidors}:

\begin{Shaded}
\begin{Highlighting}[]
\ExtensionTok{docker{-}compose}\NormalTok{ build}
\ExtensionTok{docker{-}compose}\NormalTok{ up}
\end{Highlighting}
\end{Shaded}
\item
  \textbf{Aplicar les migracions} (només al primer inici o després de
  canvis al model):

\begin{Shaded}
\begin{Highlighting}[]
\ExtensionTok{docker{-}compose}\NormalTok{ exec web python manage.py migrate}
\end{Highlighting}
\end{Shaded}
\item
  \textbf{Crear un superusuari} (opcional):

\begin{Shaded}
\begin{Highlighting}[]
\ExtensionTok{docker{-}compose}\NormalTok{ exec web python manage.py createsuperuser}
\end{Highlighting}
\end{Shaded}
\end{enumerate}

\hypertarget{accuxe9s-a-laplicaciuxf3}{%
\subsubsection{Accés a l'aplicació}\label{accuxe9s-a-laplicaciuxf3}}

Un cop en funcionament, l'aplicació estarà disponible a: -
\textbf{Aplicació web}: http://localhost:8000 - \textbf{Interfície
d'administració}: http://localhost:8000/admin

\hypertarget{compliment-dels-12-factors}{%
\subsection{6. Compliment dels 12
Factors}\label{compliment-dels-12-factors}}

El projecte intenta cumplir la guia dels 12 factors, a continuació se'n
fa un resum:

\begin{enumerate}
\def\labelenumi{\arabic{enumi}.}
\item
  \textbf{Base de codi}: Mantenim un repositori Git únic per tot el codi
  font del projecte.
\item
  \textbf{Dependències}: Totes les dependències estan explícitament
  declarades i aïllades mitjançant Poetry, permetent un control precís
  de les versions.
\item
  \textbf{Configuració}: Tot i que actualment no utilitzem variables
  d'entorn per a la configuració, el projecte està preparat per
  implementar-les en el futur quan sigui necessari.
\item
  \textbf{Serveis de suport}: La base de dades és tractada com un recurs
  extern vinculat, encara que actualment utilitzem SQLite.
\item
  \textbf{Construcció, publicació, execució}: El procés de desplegament
  està clarament separat en aquestes fases mitjançant Docker, tot i que
  de moment no necessitem escalar.
\item
  \textbf{Processos}: L'aplicació s'executa actualment com un únic
  procés, seguint el model de Django.
\item
  \textbf{Assignació de ports}: L'aplicació s'exposa a través d'un port
  específic definit al docker-compose.yml.
\item
  \textbf{Concurrència}: L'arquitectura està preparada per créixer, tot
  i que en l'estat actual no requereix múltiples processos.
\item
  \textbf{Descartabilitat}: Els contenidors Docker es poden iniciar i
  aturar ràpidament sense afectar la integritat del sistema.
\item
  \textbf{Paritat entre entorns}: Els entorns de desenvolupament i
  producció són el més similars possible gràcies a Docker.
\item
  \textbf{Logs}: Es tracten els logs com a fluxos d'esdeveniments, que
  es poden consultar mitjançant \texttt{docker-compose\ logs}.
\item
  \textbf{Processos d'administració}: Les tasques administratives
  s'executen com a processos únics, com demostra l'ús de
  \texttt{python\ manage.py}.
\end{enumerate}
